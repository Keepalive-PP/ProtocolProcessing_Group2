\begin{DoxyAuthor}{Authors}
\begin{DoxyItemize}
\item Antti Siirilä, 501449, \href{mailto:anjosi@utu.fi}{\tt anjosi@utu.\-fi}\item Heikki Juva\item Iiro Räsänen \end{DoxyItemize}

\end{DoxyAuthor}
\begin{DoxyDate}{Date}
13.\-5.\-2013 
\end{DoxyDate}
\hypertarget{index_s_intro}{}\section{Introduction}\label{index_s_intro}
B\-G\-P is a vital part of the Internet providing routing services between autonomous system. The goal of this projecs was to get familiar with the B\-G\-P protocol through a programming project. The programming projects contained the design and implementation of a B\-G\-P simulator. The design contains two main components the \hyperlink{classSimulation}{Simulation} and the User \hyperlink{classInterface}{Interface} (\hyperlink{namespaceUI}{U\-I}). The implementation of the \hyperlink{classSimulation}{Simulation} bases on System\-C where as the \hyperlink{namespaceUI}{U\-I} was implemeted with Python. \hypertarget{index_s_design}{}\section{Design}\label{index_s_design}
The architecture of the B\-G\-P simulator has devided in two modules\-: User \hyperlink{classInterface}{Interface} (\hyperlink{namespaceUI}{U\-I}) implemented with Python, and \hyperlink{classSimulation}{Simulation} implemented with the System\-C. The figure 1 illustrates the modules and their interconnection.  The \hyperlink{namespaceUI}{U\-I} allows the user to setup the simulation environment including the number of routers, the network topology, and the router configurations. Once the simulation configuration is setup, it is sent to the sc\-\_\-main through the T\-C\-P socket. The sc\-\_\-main is responsiple on building the simulation environment according to the configuration data received from the \hyperlink{namespaceUI}{U\-I}. Once the sc\-\_\-main has elaborated the \hyperlink{classSimulation}{Simulation}, it passes the T\-C\-P socket handle to the \hyperlink{classSimulation}{Simulation} and start the System\-C kernel. From there on the B\-G\-P simulation stars and the \hyperlink{namespaceUI}{U\-I} is able to receive simulation status over the T\-C\-P link by using a defined set of commands (\hyperlink{GUIProtocolTags_8hpp}{G\-U\-I\-Protocol\-Tags.\-hpp}). \hypertarget{index_sub_Simulation}{}\subsection{Simulation}\label{index_sub_Simulation}
The simulation module (figure 2) comprises of network nodes and their interconnection. Each network node contains the \hyperlink{classRouter}{Router} and \hyperlink{classHost}{Host} modules. The host module represents the local A\-S and can be used to send and receive I\-P packets to/from other A\-Ses.  The figure 3 defines the submodules and their relations inside the router module. The main modules are Data and Control planes, Routing table, and network Interfaces.  The number of network interfaces of the \hyperlink{classRouter}{Router} is defined to four. Three of the interfaces are dedicated for B\-G\-P peer connections and the remaining one connects to the host module. Each \hyperlink{classInterface}{Interface} has receiving and forwarding buffers. The \hyperlink{classDataPlane}{Data\-Plane} (D\-P) module handles the I\-P packet and B\-G\-P message forwarding. It connects to the receiving and forwarding buffers of all interfaces. In addition the D\-P connect to the \hyperlink{classRoutingTable}{Routing\-Table} (R\-T), which allows it to resolve routes for I\-P packets. Furhtermore, the D\-P passes all the B\-G\-P message to the \hyperlink{classControlPlane}{Control\-Plane} (C\-P) module and provides an interface for B\-G\-P message forwarding. I\-P packets are processed on bitlevel by \hyperlink{classPacketProcessor}{Packet\-Processor} submodule. The forwarding function includes the I\-P packet validation, destination resolving, T\-T\-L decrementation, and checksum recalculation processes. The C\-P has three B\-G\-P sessions for each B\-G\-P adjacent interface as submodules. \hyperlink{classControlPlane}{Control\-Plane} simply distributes the received B\-G\-P messages to the correct b\-G\-P session. The \hyperlink{classBGPSession}{B\-G\-P\-Session} module controls the B\-G\-P session states and timers. Once the peer interface of B\-G\-P session comes up, the session tries to establish a connection with the adjacent B\-G\-P router. The connection states follows the B\-G\-P specification (\href{https://en.wikipedia.org/wiki/Border_Gateway_Protocol}{\tt https\-://en.\-wikipedia.\-org/wiki/\-Border\-\_\-\-Gateway\-\_\-\-Protocol}). Once the connection is successfully established including the B\-G\-P O\-P\-E\-N message exchange, the B\-G\-P session transitions to the E\-S\-T\-A\-B\-L\-I\-S\-H\-E\-D state. In E\-S\-T\-A\-B\-L\-I\-S\-H\-E\-D state the B\-G\-P session simply sends keepalives, if required, and forwards B\-G\-P U\-P\-D\-A\-T\-E messages to the \hyperlink{classRoutingTable}{Routing\-Table} (R\-T). Whenever the hold-\/down timer expires or a N\-O\-T\-I\-F\-I\-C\-A\-T\-I\-O\-N message is received, the \hyperlink{classBGPSession}{B\-G\-P\-Session} invalidates and transition to the I\-D\-L\-E state. The \hyperlink{classRoutingTable}{Routing\-Table} module builds the B\-G\-P routing table based on the local A\-S information and the received B\-G\-P U\-P\-D\-A\-T\-E messages. R\-T listens the \hyperlink{classBGPSession}{B\-G\-P\-Session} states and updates the routing table and generates required U\-P\-D\-A\-T\-E messages in case of change in state. In addition, R\-T reads the U\-P\-D\-A\-T\-E messages from the input buffer and does the required table updates according to B\-G\-P specification and local policies. \hypertarget{index_sub_UI}{}\subsection{User Interface}\label{index_sub_UI}
\hyperlink{classSimulation}{Simulation} \hyperlink{namespaceUI}{U\-I} is based on Python 2.\-7 and Py\-Game-\/library, that handles the graphics. Py\-Game is extended with our U\-I-\/library, that is derived from another project, called Xadir, that needed simple U\-I-\/components such as buttons and panels. The U\-I-\/library was also co-\/authored by Heikki Juva, and expanded during this project with needed classes. \hyperlink{classSimulation}{Simulation} \hyperlink{namespaceUI}{U\-I} uses also Select\-Dialog-\/class by Aleksi Torhamo, also originally created for Xadir-\/project. The Ez\-Text-\/library by Anonymous author is found in \href{http://www.pygame.org/project-EzText-920-.html}{\tt http\-://www.\-pygame.\-org/project-\/\-Ez\-Text-\/920-\/.\-html}. This library is used for text-\/input, and extended for better support of selection one textfield and input-\/characters.

The \hyperlink{classSimulation}{Simulation} U\-I-\/library defines two types of classes, U\-I-\/ and Data-\/classes. U\-I-\/class defines strictly U\-I-\/related class, for example the Router\-Model-\/class that is used to represent selected router and displays information about the given router. Data-\/classes are defined in \hyperlink{classSimulation}{Simulation} \hyperlink{namespaceUI}{U\-I} to represent the objects in System\-C-\/simulation. This makes it simple to create simulation objects, such as Routers, Interfaces and so on in \hyperlink{classSimulation}{Simulation} \hyperlink{namespaceUI}{U\-I} before the System\-C-\/simulation is started. After the System\-C simulation is started, \hyperlink{classSimulation}{Simulation} \hyperlink{namespaceUI}{U\-I} objects are periodically synced using the socket-\/connection. Good example of this is the routing tables, that are read from router in System\-C simulation and given for corresponding \hyperlink{classSimulation}{Simulation} \hyperlink{namespaceUI}{U\-I} router. Then the selectdialogs in \hyperlink{classSimulation}{Simulation} \hyperlink{namespaceUI}{U\-I} read the tables from objects defined in Python.

Because the simulation has to be widely configurable and at the same time user need information about all of the aspects of simulation, such as routers, packets, network and routing tables, we decided to use console-\/type input system to setup the simulation and control simulation when running. Most of the advanced configs about routers and connections are available from console, such as setting router prefix. More simple things, such as creating routers and connecting routers together can be done with mouse. You can check available console commands by typing command \char`\"{}help\char`\"{} inside console.

When user has defined the desired simulation options, simulation can be started by clicking \char`\"{}\-Start\char`\"{}. This sends the configuration to System\-C-\/simulation by socket-\/connection and starts updating the routing tables. Now user can send packets between routers, disable connections from console and follow the structure of routing tables when modifying the network topology. \hypertarget{index_s_conclusion}{}\section{Conclusion}\label{index_s_conclusion}
In this programming exercise, a B\-G\-P network simulator was designed, and implemented using Python and System\-C. The B\-G\-P cold start and table update processes were successfully simulated for start topologies. The I\-P packet processesing was implemented on bitlevel as specified in R\-F\-C 1812. The I\-P packet routing was also simulated by sending test packets between A\-Ses. \hypertarget{index_sub_futureImprovements}{}\subsection{Future Improvements}\label{index_sub_futureImprovements}
Some of the basic functionality such as withdrawing a route and avoiding routing loops was not successfully simulated due to some bugs in the R\-T module. The simulation of the previously mentioned functions of B\-G\-P would require a thorough debugging of R\-T and perhaps a redesign of some of its components. In addition, implementing some experimental B\-G\-P functionalites such as security improvments would be interesting to simulate in the future. \hypertarget{index_sub_whatwelearned}{}\subsection{What We Learned}\label{index_sub_whatwelearned}
The goal of this project was to study B\-G\-P-\/protocol and its functions as an inter domain routing protocol of the Internet. The method was to build software that simulates those B\-G\-P functions. We spent several weeks for researching B\-P\-G-\/protocol standards and identifying the required aspects for building a working B\-G\-P-\/simulator. As an outcome of the researching process, we gained a strong knowledge of network protocols in general, as well as how those different protocols intervene in currently used B\-G\-P-\/networks. In addition, we adopted a strong knowledge on router architectures and configurations, and in I\-P-\/packet processing on the network layer (bit level processing). The information extracted during the research phase was put directly into the use in the B\-G\-P-\/simulator's design phase. We had to define the datatypes, classes and logic to store, transfer, and process all the data on B\-G\-P-\/simulator. Furthermore, we had to build a working and usable user interface for users to configure and control the simulation.

All in all, this project provided a perfect way to get at the deep end of B\-G\-P-\/protocol and Internet routing on B\-G\-P-\/level. Although it took relatively long time to get the simulation software finished and running, we think that there is no better way to gain as strong practical knowledge on the network protocols and protocol processing as we have been able to do during this project. This extremely interesting project has been an outstanding learning experience and will for sure give us strength and motivation for our future projects. We also have a strong will to continue the development of the B\-G\-P-\/simulator in the direction mentioned in the previous chapter. 